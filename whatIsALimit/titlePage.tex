\documentclass{ximera}

\input{../preamble.tex}

\title{What is a limit?}

\begin{document}

\begin{abstract}
%Stuff can go here later if we want!
\end{abstract}

\maketitle

\begin{sectionOutcomes}
\textbf{Goals of these lessons}: After completing this section, students should be able to do the following.

\begin{itemize}
	%\item Consider function values nearer and nearer to a given input value.
	\item Understand the concept of a limit.
    \item Use limits to help understand local behavior of functions.
	\item Calculate limits from a graph (or state that the limit does not exist).
	\item Understand possible issues when estimating limits using
          nearby values.
	\item Define a one-sided limit.
	\item Explain the relationship between one-sided and two-sided limits.
	\item Distinguish between limit values and function values.
	\item Identify when a limit does not exist.\\
\end{itemize}
\end{sectionOutcomes}

\phantom{text}%%% Making vertical spaces


%---Prereq notes/links----------%
\textbf{Skills you may want to brush up on first}: To be ready to work
through these lessons, you may need to review the following trigonometry and algebra topics.
\begin{itemize}
    \item \link[Understanding functions]{https://ximera.osu.edu/course/ColoradoStateMath/Math160Prerequisite/master/understandingFunctions/digInForEachInputExactlyOneOutput}.%%
    \item Domain
\end{itemize}

\phantom{text}%%% Making vertical spaces

\textbf{Warning}: If you are in canvas and you click on the links above, they will redirect you
to the Algebra and Trigonometry page in Ximera. You will need to
go into your 'modules' page and go back into the Exam 1 pages again
when you feel you are ready to move to the Calculus course content.


\end{document}